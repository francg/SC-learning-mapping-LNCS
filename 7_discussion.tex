% A significant portion of the articles dates back to 2005, this is probably connected to the momentum of the freshly introduced term of \textit{smart} city back in those years, like mentioned in section \ref{sec:intro}.
% From the articles screened, some common concepts and research set-ups emerged.
% For example several case studies involves schools, students and mobile devices.

% A more technological-based approach, for example using custom-built hardware and prototypes, is almost never used.
% From the methodological point of view, there seem to be a lack of Design Science\cite{hevner_three_2007} based approach, which would allow to focus more on the technology that fits best the desired learning objectives.

\subsection*{Smart city learning as a multi-faceted concept}
As for the concept of smart city, smart city learning is an overloaded concept. Though the term smart city learning has gained popularity, it seems there is no general understanding of the concept. As for many other terms, it does not seem to make sense to aim at a precise definition since its power is in creating an overlying umbrella.


\subsection*{The complexity of smart city learning}
Smart city learning is emerging as a rather complex endeavor, that is challenging the way we think about technology-enhanced learning. Complexity of smart city learning is emerging along multiple dimensions.
\begin{itemize}
\item \textit{Stakeholders}. Though most of the research that we have identified is actually initiated in school context, the learning process generally involves the cooperation of different stakeholders, e.g. public sectors, other citizens, domain experts. The cooperation of people with different interests and competencies creates a richer learning space, but at the same time it leads to learning processes that are more difficult to shape and to coordinate.
\item \textit{Activities}. Most of the examples of smart city learning presented in the literature include a combination of activities, often bridging formal and informal learning. These might include, for example, data collection and generation of data in-situ, co-located and distributed processes of sense-making, participation in complex city processes, like urban planning, and sharing of knowledge within different communities. All these activities require different competencies, skills, and assessment criteria.
\item \textit{Technologies}. Smart city learning is often enhanced by different technologies, either dedicated or general purpose, like for example social media and sharing platforms. More than large institutional systems, like e.g. dedicated learning management systems, the field seems to be characterized by a tailored adoption of multiple lightweight systems. In addition, the success of the learning experience is relying on the availability of a technical infrastructure to promote communication.
\end{itemize}

As a consequence of this complexity we need to re-think our research methods and design processes to meet the specific challenges of this new domain. Smart city learning is happening in complex eco-systems that require new theoretical approaches, multidisciplinary approaches, and new pedagogics. A literacy of participation needs to be developed.


\subsection*{Unexplored technical opportunities}
In the mapping we have identified a number of interesting concepts and technological solutions. However, we have also identified two technical opportunities that are, somehow unexpectedly, not yet used.
\begin{itemize}
\item The proposed solutions are mostly relying on mobile technologies, e.g. phones. Novel interaction modalities, e.g. interactive objects and the Internet of Things (IoT), are not fully exploited. Some works exploit the possibility to tag objects with RFID for situated access to information \cite{kashtan_outdoors_2013}. These works show how novel technological solutions could be exploited to situate more learning activities. Novel interaction modalities might also support interaction with new categories of users, e.g. the elderly.
\item \textit{Big data, real-time data, small-data}. Smart cities are characterized by technical infrastructures that produce rich datasets, e.g. about mobility, energy consumption, environmental data. The solutions that are currently proposed in the literature are not fully exploiting the possibilities offered by city-related data. In the wider research area of Human Computer Interaction, the use of city data is for example used to increase awareness of environmental issues, see e.g. \cite{dourish_hci_2010}. At the same time, research in the area of the quantified-self is taking advantage of data generated by each individuals to promote e.g. sustainable behavior. Bringing together the quantified city with the quantified self might lead to new interesting opportunities for learning, especially in the area of sustainable behavior. Of course, the use of data, especially big data, comes with a number of risks and ethical concerns, but investigating its use for social innovation and learning seems to be a promising area of future research.
\end{itemize}
