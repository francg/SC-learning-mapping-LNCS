After the search and collection phase, articles meta-data were exported to a spreadsheet for screening and selection of relevant topics for the study.
All the titles, and if necessary the abstracts, were read to determine which articles to include in the study.

The PRISMA Statement for Reporting Systematic Reviews and Meta-Analyses\cite{liberati_prisma_2009} was used to guide and structure the criteria of inclusion/exclusion.
More precisely the authors used report eligibility and study eligibility criteria.

Study eligibility criteria are likely to include the populations, interventions, comparators, outcomes, and study designs of interest, as well as other study-specific elements, such as specifying a minimum length of follow-up. Authors should state whether studies will be excluded because they do not include (or report) specific outcomes to help readers ascertain whether the systematic review may be biased as a consequence of selective reporting\cite{liberati_prisma_2009}.

Report eligibility criteria are likely to include language of publication, publication status (e.g., inclusion of unpublished material and abstracts), and year of publication. Inclusion or not of non-English language literature, unpublished data, or older data can influence the effect estimates in meta-analyses. Caution may need to be exercised in including all identified studies due to potential differences in the risk of bias such as, for example, selective reporting in abstracts\cite{liberati_prisma_2009}.

To follow the Report and Study eligibility criteria adopted for this work.

\medskip

\textbf{Report Eligibility}
\begin{enumerate}
\item Publications should be in English;
\item Articles should be published on peer-reviewed journals, international conferences or as book chapters;
\item Year of publication should be between 2005 and 2015;
\item Publications must have an abstract.
\end{enumerate}

\medskip

\textbf{Study Eligibility}
\begin{enumerate}
\item The city perspective must comply with the definition provided in section \ref{subsec:definition}: the concept of city as a whole, either in the urban or citizen perspective, must be present;
\item If the object of research is a single community the study must not be limited to any urban area in the city or related to any of the infrastructure networks that permeates the city (streets, water and power lines, etc);
\item The learning factor should be present;
\item If the environment is limited to a specific context, there should be no constraints on categories of citizens that are involved or take advantage of the research;
\item The use of technology should be present and mentioned in the abstract.
\end{enumerate}

The inclusion/exclusion screening was initially performed by both authors independently on the first 100 articles. On a total of 16 articles there was disagreement, and a specific discussion on the abstract was needed to reach a final agreed decision of inclusion or exclusion.

This process was helpful to discuss, clarify and refine the criteria of inclusion/exclusion.

The following step consisted of another independent screening of 100 articles, this time the authors disagreed only on 4 articles. This two-step process helped to ensure that both the authors applied the inclusion/exclusion criteria in the same way, and allowed for the rest of the articles to be divided between the authors for the inclusion decision. Each author decided independently for inclusion/exclusion for the 50\% of the remaining articles.

The total number of included articles is 54, while the articles excluded after title/abstract screening are 1447.

No articles were included from manual search of conference proceedings.