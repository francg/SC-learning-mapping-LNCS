Technology in smart cities is essential and considered as a supporting backbone\cite{giovannella_smart_2014}.
The role of technology in smart cities has been widely recognized and addressed, however there seems to be no established field of research that connects smart cities to learning.

This work is motivated by the quest for a clear overview of existing research related to learning in smart cities.


\subsection{What do we consider as ``smart city learning''?} \label{subsec:definition}
Some of the studies that are situated in smart cities and also present a learning component take place in confined communities and facilities, like the elderly living in retirement houses or patients in hospitals. Even if these scenarios are physically situated in a (smart) city, they remain relevant and valid even if the smart city component is removed from the research context.

The \textit{smart city} term seems to be often attached to research works where it is not a central or absolutely essential element.

To avoid articles not relevant for the purpose of this work, the authors decided that the boundaries that define the adopted research scope on smart cities are dependent by two factors:

\begin{itemize}
\item The social perspective, which defines the people affected and should not be constrained by any particular bound. Every citizen can be involved.
\item The urban perspective, which includes the city as an urban space and it is not confined to any particular facility or environment that can be also found outside the smart city context.
\end{itemize}

A significant scenario should include at least one of the two factors. Here are some examples of scenarios:

\begin{enumerate}
\item Students collecting sensor data on their commute path to school or moving around the city. Data is then aggregated and presented to the community to facilitate reflection, learning and to stimulate sustainable and safer behaviors.
\item Citizens collecting energy consumption data in their house, which is then aggregated to create a energy consumption map for the whole city. Looking at the map, citizens can discover interesting patterns and reflect on the margin of improvement for their houses.
\item Bikes used for bike sharing services can be instrumented to collect air pollution and other sensor data. Cyclists around the city can provide a detailed and constantly updated sensor-map that can stimulate citizens to adopt more sustainable and efficient mobility patterns.
\end{enumerate}

All the three scenarios proposed are relevant for the smart city learning research scope defined above.

The first scenario works only within a defined community of citizens, but they are displaced in the entire urban environment of a smart city.

In the second scenario the space is confined into individual apartments and houses, but every citizen can be potentially involved. The data is also aggregated and interpreted at a city-wide level.

The third scenario combine both the social and urban perspective: there is no specific category of citizens being addressed and the relevant urban space is located in the city as a whole.


\subsection{Research Questions}
The research questions addressed are:

\begin{itemize}
\item \textbf{RQ1}: Which are the most common scenarios of application, usage settings and learning contexts within technology-enhanced smart city learning research?
\item \textbf{RQ2}: Is there any characteristic publication pattern?
\item \textbf{RQ3}: Which kind of features and patterns characterize the technological applications?
\item \textbf{RQ4}: Which learning theories and approaches are most commonly used?
\item \textbf{RQ5}: What type of research is performed and which methods are used?
\end{itemize}

