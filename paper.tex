\documentclass[runningheads,a4paper]{llncs}

\usepackage[american]{babel}

\usepackage{graphicx}

%extended enumerate, such as \begin{compactenum}
\usepackage{paralist}

%put figures inside a text
%\usepackage{picins}
%use
%\piccaptioninside
%\piccaption{...}
%\parpic[r]{\includegraphics ...}
%Text...

%Sorts the citations in the brackets
%\usepackage{cite}

%for easy quotations: \enquote{text}
\usepackage{csquotes}

\usepackage[T1]{fontenc}

%enable margin kerning
\usepackage{microtype}

%better font, similar to the default springer font
\usepackage[%
rm={oldstyle=false,proportional=true},%
sf={oldstyle=false,proportional=true},%
tt={oldstyle=false,proportional=true,variable=true},%
qt=false%
]{cfr-lm}
%
%if more space is needed, exchange cfr-lm by mathptmx
%\usepackage{mathptmx}

%for demonstration purposes only
\usepackage[math]{blindtext}

\usepackage[
%pdfauthor={},
%pdfsubject={},
%pdftitle={},
%pdfkeywords={},
bookmarks=false,
breaklinks=true,
colorlinks=true,
linkcolor=black,
citecolor=black,
urlcolor=black,
%pdfstartpage=19,
pdfpagelayout=SinglePage
]{hyperref}
%enables correct jumping to figures when referencing
\usepackage[all]{hypcap}

\usepackage{url}

\usepackage[table,xcdraw]{xcolor}
\usepackage{footnote}
\authorrunning{Francesco Gianni et al.}
\titlerunning{Technology Support in SC Learning: a Systematic Mapping of the Literature}


\usepackage[capitalise,nameinlink]{cleveref}
%Nice formats for \cref
\crefname{section}{Sect.}{Sect.}
\Crefname{section}{Section}{Sections}
\crefname{figure}{Fig.}{Fig.}
\Crefname{figure}{Figure}{Figures}

\usepackage{xspace}
%\newcommand{\eg}{e.\,g.\xspace}
%\newcommand{\ie}{i.\,e.\xspace}
\newcommand{\eg}{e.\,g.,\ }
\newcommand{\ie}{i.\,e.,\ }

%introduce \powerset - hint by http://matheplanet.com/matheplanet/nuke/html/viewtopic.php?topic=136492&post_id=997377
\DeclareFontFamily{U}{MnSymbolC}{}
\DeclareSymbolFont{MnSyC}{U}{MnSymbolC}{m}{n}
\DeclareFontShape{U}{MnSymbolC}{m}{n}{
    <-6>  MnSymbolC5
   <6-7>  MnSymbolC6
   <7-8>  MnSymbolC7
   <8-9>  MnSymbolC8
   <9-10> MnSymbolC9
  <10-12> MnSymbolC10
  <12->   MnSymbolC12%
}{}
\DeclareMathSymbol{\powerset}{\mathord}{MnSyC}{180}

%improve wrapping of URLs - hint by http://tex.stackexchange.com/a/10419/9075
\makeatletter
\g@addto@macro{\UrlBreaks}{\UrlOrds}
\makeatother

% correct bad hyphenation here
\hyphenation{op-tical net-works semi-conduc-tor}

\begin{document}

%Works on MiKTeX only
%hint by http://goemonx.blogspot.de/2012/01/pdflatex-ligaturen-und-copynpaste.html
%also http://tex.stackexchange.com/questions/4397/make-ligatures-in-linux-libertine-copyable-and-searchable
%This allows a copy'n'paste of the text from the paper
\input glyphtounicode.tex
\pdfgentounicode=1

\title{Technology Support in Smart City Learning:\\ a Systematic Mapping of the Literature}
%If Title is too long, use \titlerunning
%\titlerunning{Short Title}

%Single insitute
\author{Francesco Gianni \and Monica Divitini\\
francesco.gianni@idi.ntnu.no, monica.divitini@idi.ntnu.no}
%If there are too many authors, use \authorrunning
%\authorrunning{First Author et al.}
\institute{Norwegian Univesity of Science and Technology\\
Department of Computer and Information Science\\
Trondheim, Norway}

%Multiple insitutes
%Currently disabled
%
\iffalse
%Multiple institutes are typeset as follows:
\author{Firstname Lastname\inst{1} \and Firstname Lastname\inst{2} }
%If there are too many authors, use \authorrunning
%\authorrunning{First Author et al.}

\institute{
Insitute 1\\
\email{...}\and
Insitute 2\\
\email{...}
}
\fi
			
\maketitle

\begin{abstract}
  Smart cities are a popular and recognized research topic. In urban spaces, the learning factor is an important component for citizens and local communities. This paper presents a systematic mapping of the literature on smart city learning, with focus on how technology is used to enhance smart city learning.
The goal is to map the state of the art and to identify gaps in current research that can prompt new research in this area.

Articles were collected from various online databases and relevant journal publications, selected according to defined inclusion/exclusion criteria. Abstracts were coded based on a number of criteria, including e.g. learning goal, used technology, and theoretical approach. Following the coding process results were analyzed to identify themes. 

In the paper we shed light on the current understanding of smart city learning by (i) Identifying common scenarios and learning settings; (ii) publication patterns; (iii) technical features in the supporting technology; (iv) learning theories and approaches that are mostly used; and (v) adopted type of research and research methods.

The mapping shows that the concept of smart city learning is growing in popularity, with increasing number of publications in this area in the last years. However, the field is rather fragmented, with very different understanding of the concept.  Smart city learning is also emerging as a very complex form of learning, with different stakeholders, learning activities, and technological solutions combined in rich eco-systems. The mapping also points out two largely unexplored areas of technological support, namely the Internet of Things (IoT) and the use of city-related data.

\end{abstract}

%%\keywords{...}

%%%%%%%%%%%%%%%%%%%%%%%%%%%%%%%%%%%%%%%%%%%%%%%%%%%%%%%%%%%%%%%%%%%%%%%%%%%%%%%
\section{Introduction}\label{sec:intro}
The concept of smart-city has been used in many different contexts and is associated with distinctive and innovative aspects that are often quite different. Big diversities are observed on the reasons \textit{why} different cities are defined as \textit{smart}.

This situation is the consequence of the lack of a clear and recognized definition of smart city.

In attempting to pin down what is smart about the smart city, one finds that it involve quite a diverse range of things (information technology, business innovation, governance, communities and sustainability). It can also be suggested that the label itself often makes certain assumptions about the relationship between these things, for example regarding consensus and balance \cite{hollands_will_2008}.

Komninos \cite{komninos_intelligent_2002}, in his attempt to delineate the intelligent city, (perhaps the concept most closely related to the smart city), cites four possible meanings:

\begin{enumerate}
\item The application of a wide range of electronic and digital applications to communities and cities.
\item The use of information technology to transform life and work.
\item The meaning of intelligent or smart as embedded information and communication technologies.
\item The spatial territories that bring ICTs and people together to enhance innovation, learning, knowledge and problem solving.
\end{enumerate}

Overall then, Komninos \cite{komninos_intelligent_2002} sees intelligent (smart) cities as ``\textit{territories with high capacity for learning and innovation, which is built\textendash in the creativity of their population, their institutions of knowledge creation, and their digital infrastructure for communication and knowledge management}''.

The adjective ``smart'' began to gain a increasingly notoriety between 2005 and 2007, when it started to be used to denote a sort of dream-city, i.e. a complex and optimized environment, or eco-system, where it could be desirable to live. It appeared immediately clear that the adjective smart was intended to go well beyond the meaning intelligent and/or to emphasize the use of ICT and digital technologies \cite{giovannella_smart_2014-1}.

For this reason the authors chose to limit the search to articles published from 2005.

Smart cities are also a powerful and recognized ecosystem for learning. Smart city learning aim to support the improvement of all key factors contributing to the regional competitiveness: mobility, environment, people, quality of life and governance \cite{hollands_will_2008}. The approach is aimed at optimizing resource consumption and saving time improving flows of people, goods and data\footnote{\url{http://www.mifav.uniroma2.it/inevent/events/sclo/}}.

Education in this context is pursued as a bottom-up process, where person and places are central. Smartness from a learning perspective exists both in the ambient data collected and among the communities that exists within a city.

The separation between student and teacher fades out. Their role will be content or situation dependent: everybody will be a learner and the relation between persons will get a bigger role.

From the learning perspective, smart cities can be seen as an independent \textit{learning actor} that behaves like an autonomous entity which adapt itself in an evolving environment.

Despite this, in this paper we focus on smart cities as a place where citizens learn smart-behaviors.

This scenario can involve traditional education which happens in facilities like schools and universities. The goal of this work is instead more oriented towards \textit{lifelong learning}, defined as the continuous build of skills to adapt and collaborate in dynamic ecosystems like smart cities.

\section{Motivation and Research Questions}
Technology in smart cities is essential and considered as a supporting backbone\cite{giovannella_smart_2014}.
The role of technology in smart cities has been widely recognized and addressed, however there seems to be no established field of research that connects smart cities to learning.

This work is motivated by the quest for a clear overview of existing research related to learning in smart cities.


\subsection{What do we consider as ``smart city learning''?} \label{subsec:definition}
Some of the studies that are situated in smart cities and also present a learning component take place in confined communities and facilities, like the elderly living in retirement houses or patients in hospitals. Even if these scenarios are physically situated in a (smart) city, they remain relevant and valid even if the smart city component is removed from the research context.

The \textit{smart city} term seems to be often attached to research works where it is not a central or absolutely essential element.

To avoid articles not relevant for the purpose of this work, the authors decided that the boundaries that define the adopted research scope on smart cities are dependent by two factors:

\begin{itemize}
\item The social perspective, which defines the people affected and should not be constrained by any particular bound. Every citizen can be involved.
\item The urban perspective, which includes the city as an urban space and it is not confined to any particular facility or environment that can be also found outside the smart city context.
\end{itemize}

A significant scenario should include at least one of the two factors. Here are some examples of scenarios:

\begin{enumerate}
\item Students collecting sensor data on their commute path to school or moving around the city. Data is then aggregated and presented to the community to facilitate reflection, learning and to stimulate sustainable and safer behaviors.
\item Citizens collecting energy consumption data in their house, which is then aggregated to create a energy consumption map for the whole city. Looking at the map, citizens can discover interesting patterns and reflect on the margin of improvement for their houses.
\item Bikes used for bike sharing services can be instrumented to collect air pollution and other sensor data. Cyclists around the city can provide a detailed and constantly updated sensor-map that can stimulate citizens to adopt more sustainable and efficient mobility patterns.
\end{enumerate}

All the three scenarios proposed are relevant for the smart city learning research scope defined above.

The first scenario works only within a defined community of citizens, but they are displaced in the entire urban environment of a smart city.

In the second scenario the space is confined into individual apartments and houses, but every citizen can be potentially involved. The data is also aggregated and interpreted at a city-wide level.

The third scenario combine both the social and urban perspective: there is no specific category of citizens being addressed and the relevant urban space is located in the city as a whole.


\subsection{Research Questions}
The research questions addressed are:

\begin{itemize}
\item \textbf{RQ1}: Which are the most common scenarios of application, usage settings and learning contexts within technology-enhanced smart city learning research?
\item \textbf{RQ2}: Is there any characteristic publication pattern?
\item \textbf{RQ3}: Which kind of features and patterns characterize the technological applications?
\item \textbf{RQ4}: Which learning theories and approaches are most commonly used?
\item \textbf{RQ5}: What type of research is performed and which methods are used?
\end{itemize}



\section{Data Sources and Search}
\subsection{Data Sources}
The articles were searched and collected using three different approaches:

\begin{enumerate}
\item keyword based search on different online databases
\item manual screening of selected conference proceedings
\item manual screening of selected special issues of journals
\end{enumerate}

The following online databases were used for the keyword based search: ISI Web of Science\footnote{https://apps.webofknowledge.com/}, ACM digital library\footnote{https://dl.acm.org/}, Elsevier - ScienceDirect\footnote{https://www.sciencedirect.com/}, Elsevier - Scopus\footnote{http://www.scopus.com/}, IEEE Xplore\footnote{http://ieeexplore.ieee.org/}.

\medskip

The following conference proceedings were searched for relevant articles:
\begin{itemize}
\item \textbf{CSCW} Computer-Supported Cooperative Work and Social Computing\footnote{http://cscw.acm.org/}
\item \textbf{CHI} Conference on Human Factors in Computing Systems\footnote{http://chiYYYY.acm.org/}
\item \textbf{EC-TEL} European Conference on Technology Enhanced Learning\footnote{http://www.ec-tel.eu/}
\item \textbf{AMI} International Joint Conference on Ambient Intelligence\footnote{http://www.ami-conferences.org/}
\item \textbf{C\&T} International Conference on Communities and Technologies\footnote{http://comtech.community/}
\end{itemize}

The following journal issues were searched for relevant articles:
\begin{itemize}
\item \textbf{IJDLDC} International Journal of Digital Literacy and Digital Competence vol. 3 n. 4 - Special Issue on \textit{``Smart City Learning, literacy and Competences''}\footnote{http://www.igi-global.com/journal/international-journal-digital-literacy-digital/1170}
\item \textbf{IxD\&A} Interaction Design and Architecture(s), vol. 16 (part I)\footnote{\url{http://www.mifav.uniroma2.it/inevent/events/idea2010/index.php?s=10&a=10&link=ToC_16_P}}, vol. 17 (part II)\footnote{\url{http://www.mifav.uniroma2.it/inevent/events/idea2010/index.php?s=10&a=10&link=ToC_17_P}} - Special Issue on \textit{``Smart City Learning - Visions and practical Implementations: toward Horizon 2020''} 
\end{itemize}

\subsection{Search and Keywords}
The keywords selection process was driven by the PICO framework. PICO helps to develop a comprehensive set of search keywords for quantitative research terms according to: Population, Intervention or Exposure (PECO), Comparison, Outcomes\cite{schardt_utilization_2007}.

Initially, keywords for all the sections of the framework were selected, but the authors decided later on to relax some constraints in order to avoid missing some possible relevant articles. A \textit{context} section was also added to the schema.

Table \ref{table:keywords} shows the PICO(C) structure with associated keywords.

%% table of keywords here
\begin{table}[ht]
\renewcommand{\arraystretch}{1.3}
\centering
\caption{PICO(C) Driven Keywords Framing}
\label{table:keywords}
\begin{tabular}{|
>{\columncolor[HTML]{EFEFEF}}l |l|}
\hline
\textbf{Population} & \textit{-} \\ \hline
\textbf{Intervention} & \textit{learning} \\ \hline
\textbf{Comparison} & \textit{-} \\ \hline
\textbf{Outcome} & \textit{\begin{tabular}[c]{@{}l@{}}participation, collaboration, reflection,\\ awareness\end{tabular}} \\ \hline
\textbf{Context} & \textit{\begin{tabular}[c]{@{}l@{}}cities, smart city, urban, connected city,\\ intelligent city, digital city\end{tabular}} \\ \hline
\end{tabular}
\end{table}

A pilot search was conducted on some of the online databases in order to refine the keywords and find a search query that could be adapted and used in all the different online databases.

The final search query used is reported in Table \ref{table:query}.

%% search query
\begin{table}[ht]
\renewcommand{\arraystretch}{1.3}
\centering
\caption{Search Query}
\label{table:query}
\begin{tabular}{c|c}
\rowcolor[HTML]{EFEFEF} 
\textbf{Context} & \begin{tabular}[c]{@{}c@{}}(cities OR "smart city" OR urban\\ OR "connected city" OR "intelligent city"\\ OR "digital city")\end{tabular} \\
 & \textbf{AND} \\
\rowcolor[HTML]{EFEFEF} 
\textbf{Intervention} & ("learning") \\
 & \textbf{AND} \\
\rowcolor[HTML]{EFEFEF} 
\textbf{Outcome} & \begin{tabular}[c]{@{}c@{}}(participation OR collaboration\\ OR reflection OR awareness)\end{tabular}
\end{tabular}
\end{table}

Different online databases offer different levels of search functionalities and details when going to use a complex query that possibly involves several keywords and fields (title, abstract, etc.).

Some of the difficulties encountered were:
\begin{itemize}
\item limit on the number of keywords that can be used;
\item limit on the fields where the search can be performed, search on title AND abstract not always possible;
\item no precise and direct control on the target search fields, keywords could be only searched on a preset aggregation of fields like title, abstract, article keywords;
\item different ways of coding the same logic expression, the same search string couldn't be reused on different databases;
\item different formats of the result set, in some cases was possible to batch-download the results, otherwise results had been scraped using Zotero\footnote{https://www.zotero.org/} browser integration.
\end{itemize}

The keywords were searched on title and abstract when possible, otherwise only the abstract was used. In Table \ref{table:result_1} the size of the result sets for each online database are outlined.

The articles collected were imported in a Zotero library, and duplicates were manually removed.

%% table of result count
\begin{savenotes}
\begin{table}[ht]
\setlength{\tabcolsep}{8pt}
\centering
\caption{Result Set for online databases before duplicates removal}
\label{table:result_1}
\begin{tabular}{cccccc|c}
 & \cellcolor[HTML]{EFEFEF}\textbf{ISI} & \cellcolor[HTML]{EFEFEF}\textbf{ACM} & \cellcolor[HTML]{EFEFEF}\textbf{ScienceDirect} & \cellcolor[HTML]{EFEFEF}\textbf{Scopus} & \cellcolor[HTML]{EFEFEF}\textbf{IEEE} & \textbf{TOT} \\
\textit{n} & 938 & 35 & 162 & 1022 & 42 & 2199 \\
\textit{field} & topic\footnote{Topic fields include Titles, Abstracts, Keywords and Indexing fields such as Systematics, Taxonomic Terms and Descriptors} & abstract & abstract & abstract & abstract & 
\end{tabular}
\end{table}
\end{savenotes}

%% table no duplicates
\begin{table}[ht]
\setlength{\tabcolsep}{8pt}
\centering
\caption{Final Result Set without duplicates and selected articles for coding}
\label{table:result_2}
\begin{tabular}{cccc|c}
 & \cellcolor[HTML]{EFEFEF}\textbf{Online Databases} & \cellcolor[HTML]{EFEFEF}\textbf{IJDLDC} & \cellcolor[HTML]{EFEFEF}\textbf{IxD\&A} & \textbf{TOT} \\
\textit{n (no duplicates)} & 1485 & 5 & 11 & 1501 \\
\textit{selected} & 43 & 2 & 9 & 54
\end{tabular}
\end{table}

\section{Screening of Papers}
After the search and collection phase, articles meta-data were exported to a spreadsheet for screening and selection of relevant topics for the study.
All the titles, and if necessary the abstracts, were read to determine which articles to include in the study.

The PRISMA Statement for Reporting Systematic Reviews and Meta-Analyses\cite{liberati_prisma_2009} was used to guide and structure the criteria of inclusion/exclusion.
More precisely the authors used report eligibility and study eligibility criteria.

Study eligibility criteria are likely to include the populations, interventions, comparators, outcomes, and study designs of interest, as well as other study-specific elements, such as specifying a minimum length of follow-up. Authors should state whether studies will be excluded because they do not include (or report) specific outcomes to help readers ascertain whether the systematic review may be biased as a consequence of selective reporting\cite{liberati_prisma_2009}.

Report eligibility criteria are likely to include language of publication, publication status (e.g., inclusion of unpublished material and abstracts), and year of publication. Inclusion or not of non-English language literature, unpublished data, or older data can influence the effect estimates in meta-analyses. Caution may need to be exercised in including all identified studies due to potential differences in the risk of bias such as, for example, selective reporting in abstracts\cite{liberati_prisma_2009}.

To follow the Report and Study eligibility criteria adopted for this work.

\medskip

\textbf{Report Eligibility}
\begin{enumerate}
\item Publications should be in English;
\item Articles should be published on peer-reviewed journals, international conferences or as book chapters;
\item Year of publication should be between 2005 and 2015;
\item Publications must have an abstract.
\end{enumerate}

\medskip

\textbf{Study Eligibility}
\begin{enumerate}
\item The city perspective must comply with the definition provided in section \ref{subsec:definition}: the concept of city as a whole, either in the urban or citizen perspective, must be present;
\item If the object of research is a single community the study must not be limited to any urban area in the city or should be related to one of the infrastructure networks that permeates the city (streets, water and power lines, etc);
\item The learning factor should be present;
\item If the environment is limited to a specific context, there should be no constraints on categories of citizens that are involved or take advantage of the research;
\item The use of technology should be present and mentioned in the abstract.
\end{enumerate}

The inclusion/exclusion screening was initially performed by both authors independently on the first 100 articles. On a total of 16 articles there was disagreement, and a specific discussion on the abstract was needed to reach a final decision of inclusion or exclusion.

This process was helpful to discuss, clarify and refine the criteria of inclusion/exclusion.

The following step consisted of another independent screening of 100 articles, this time the authors disagreed only on 4 articles. This two-step process helped to ensure that both the authors applied the inclusion/exclusion criteria in the same way, and allowed for the rest of the articles to be divided between the authors for the inclusion decision. Each author decided independently for inclusion/exclusion for the 50\% of the remaining articles.

The total number of included articles is 54, while the articles excluded after title/abstract screening are 1447.

No articles were included from manual search of conference proceedings.

\section{Classification and Coding}
A first classification structure was drafted and used by one of the authors to code the first 20 abstracts. The coding of these abstracts and the classification were then discussed and revised by both authors.

The classification structure was created in Nvivo\footnote{http://www.qsrinternational.com/} and was organized in two nested levels.

The authors decided to use an \textit{emerging} approach when working on the categories: new elements were dynamically added during the coding process.

The coding process itself consisted in reading the abstract and \textit{tagging} relevant chunks of text with one or more categories.

The number of categories that could be correlated to any single publication was strictly connected with the richness and accuracy of the abstract. More information-rich and structured abstracts were tagged with more categories than shorter ones.

\section{Results and Findings}
\subsection*{RQ1: scenarios and learning contexts}

The main research areas seems to be connected to schools and governance.
This is confirmed by the fact that the target population of the studies and/or the community affected by the learning process is often the students.
This result can be correlated to the finding that more than 68\% of the articles present the concept of the city as a place where learning happens. This point of view is quite distinct to the more engaging concept of actively living the city learning behaviours and generating knowledge, which can be considered a lifelong learning experience to improve the quality of urban living.


\subsection*{RQ2: publication pattern}

Research on smart-city learning gained approval and popularity quite constantly during the years. A relatively important amount of articles dates back to 2005, starting year of the chosen interval. From 2006 a general increase in publications can be noted till 2014. The year 2015 was excluded from this statistic since not all research on the topic was yet published when articles were collected.

\begin{figure}[tbh]
\centering
\includegraphics[width=9cm]{img/years}
\caption{Research publications per year.}
\label{fig:years}
\end{figure}

Selected articles are almost equally divided between international conference proceeding publications and journal or book chapters.
Publication from international journals remains the overall prevalent group.

\begin{figure}[tbh]
\centering
\includegraphics[width=9cm]{img/publication}
\caption{Types of publication.}
\label{fig:publications}
\end{figure}


\subsection*{RQ3: application of technology}

The technological pattern involved in smart-city learning is, most of the times, connected to support the learning process.
For this purpose, the use of mobile devices is the prevailing choice.

Online cooperative platforms of various types are also used in many cases: more precisely e-learning and e-government solutions were mentioned in more than one article.


\subsection*{RQ4: learning theories and approaches}

Some articles mentioned specific learning theories applied during the study. Game-based Learning and Situated Learning\cite{anderson_situated_1996} are the approach that were reported more often.
The approaches that are most often pursued are connected to various levels of collaboration and cooperation between stakeholders or within the learning community (for example among the students). Context awareness and situatedness are also mentioned in a few articles.

\subsection*{RQ5: research methods and types}
Research on smart-city learning often involves case studies oriented to perform some sort of investigation on a specific problem. Solution design or implementation studies are less common, even more rare are studies that make use of IOT, ubiquitous technologies and custom hardware prototyping.

\section{Discussion}
% A significant portion of the articles dates back to 2005, this is probably connected to the momentum of the freshly introduced term of \textit{smart} city back in those years, like mentioned in section \ref{sec:intro}.
% From the articles screened, some common concepts and research set-ups emerged.
% For example several case studies involves schools, students and mobile devices.

% A more technological-based approach, for example using custom-built hardware and prototypes, is almost never used.
% From the methodological point of view, there seem to be a lack of Design Science\cite{hevner_three_2007} based approach, which would allow to focus more on the technology that fits best the desired learning objectives.

\subsection*{Smart city learning as a multi-faceted concept}
As for the concept of smart city, smart city learning is an overloaded concept. Though the term smart city learning has gained popularity, it seems there is no general understanding of the concept. As for many other terms, it does not seem to make sense to aim at a precise definition since its power is in creating an overlying umbrella.


\subsection*{The complexity of smart city learning}
Smart city learning is emerging as a rather complex endeavor, that is challenging the way we think about technology-enhanced learning. Complexity of smart city learning is emerging along multiple dimensions.
\begin{itemize}
\item \textit{Stakeholders}. Though most of the research that we have identified is actually initiated in school context, the learning process generally involves the cooperation of different stakeholders, e.g. public sectors, other citizens, domain experts. The cooperation of people with different interests and competencies creates a richer learning space, but at the same time it leads to learning processes that are more difficult to shape and to coordinate.
\item \textit{Activities}. Most of the examples of smart city learning presented in the literature include a combination of activities, often bridging formal and informal learning. These might include, for example, data collection and generation of data in-situ, co-located and distributed processes of sense-making, participation in complex city processes, like urban planning, and sharing of knowledge within different communities. All these activities require different competencies, skills, and assessment criteria.
\item \textit{Technologies}. Smart city learning is often enhanced by different technologies, either dedicated or general purpose, like for example social media and sharing platforms. More than large institutional systems, like e.g. dedicated learning management systems, the field seems to be characterized by a tailored adoption of multiple lightweight systems. In addition, the success of the learning experience is relying on the availability of a technical infrastructure to promote communication.
\end{itemize}

As a consequence of this complexity we need to re-think our research methods and design processes to meet the specific challenges of this new domain. Smart city learning is happening in complex eco-systems that require new theoretical approaches, multidisciplinary approaches, and new pedagogics. A literacy of participation needs to be developed.


\subsection*{Unexplored technical opportunities}
In the mapping we have identified a number of interesting concepts and technological solutions. However, we have also identified two somehow unexpectedly technical opportunities that are not used yet.
\begin{itemize}
\item The proposed solutions are mostly relying on mobile technologies, e.g. phones. Novel interaction modalities, e.g. interactive objects and the Internet of Things (IoT), are not fully exploited. Some works exploit the possibility to tag objects with RFID for situated access to information\cite{kashtan_outdoors_2013}. These works show how novel technological solutions could be exploited to situate more learning activities. Novel interaction modalities might also support interaction with new categories of users, e.g. the elderly.
\item \textit{Big data, real-time data, small-data}. Smart cities are characterized by technical infrastructures that produces rich datasets, e.g. about mobility, energy consumption, environmental data. The solutions that are currently proposed in the literature are mostly not exploiting the possibilities offered by city-related data. In the wider research area of Human Computer Interaction, the use of city data is for example used to increase awareness of environmental issues, see e.g. \cite{kashtan_outdoors_2013}. At the same time, research in the area of the quantified-self is taking advantage of data generated by each individuals to promote e.g. sustainable behavior. Bringing together the quantified city with the quantified self might lead to new interesting opportunities for learning, especially in the area of sustainable behavior. Of course, the use of data, especially big data, comes with a number of risks and ethical concerns, but investigating its use for social innovation and learning seems to be a promising area of future research.
\end{itemize}


\section{Conclusions}
%insert block diagram schema from PRISMA
With this work we managed to provide an overview of the current status of research in the technology-enhanced Smart City Learning field.
The most common research patterns and methodologies were identified and challenges for future research were proposed.
The emerging scenario depict Smart City Learning as a promising field, both regarding the potential applications aimed at increasing urban quality of life and as opportunities for researchers to empower unexplored approaches and technologies.


%%%%%%%%%%%%%%%%%%%%%%%%%%%%%%%%%%%%%%%%%%%%%%%%%%%%%%%%%%%%%%%%%%%%%%%%%%%%%%%
\bibliographystyle{splncs03}
% \bibliographystyle{IEEEtran}
\bibliography{references_scl}

%%%%%%%%%%%%%%%%%%%%%%%%%%%%%%%%%%%%%%%%%%%%%%%%%%%%%%%%%%%%%%%%%%%%%%%%%%%%%%%

\end{document}
