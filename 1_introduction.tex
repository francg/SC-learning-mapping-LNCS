The concept of smart-city has been used in many different context and is associated with distinctive and innovative aspects that are often quite different. Big diversities are observed on the reasons \textit{why} different cities are defined as \textit{smart}.

This situation is the consequence of the lack of a clear and recognized definition of smart city.

In attempting to pin down what is smart about the smart city, one finds that not only does it involve quite a diverse range of things (information technology, business innovation, governance, communities and sustainability) it can also be suggested that the label itself often makes certain assumptions about the relationship between these things, for example regarding consensus and balance\cite{hollands_will_2008}.

Komninos\cite{komninos_intelligent_2002}, in his attempt to delineate the intelligent city, (perhaps the concept most closely related to the smart city), cites four possible meanings:

\begin{enumerate}
\item The application of a wide range of electronic and digital applications to communities and cities, which effectively work to conflate the term with ideas about the cyber, digital, wired, informational or knowledge\textendash based city.
\item The use of information technology to transform life and work within a region in significant and fundamental ways.
\item The meaning of intelligent or smart as embedded information and communication technologies in the city.
\item The spatial territories that bring ICTs and people together to enhance innovation, learning, knowledge and problem solving.
\end{enumerate}

Overall then, Komninos\cite{komninos_intelligent_2002} sees intelligent (smart) cities as ``\textit{territories with high capacity for learning and innovation, which is built\textendash in the creativity of their population, their institutions of knowledge creation, and their digital infrastructure for communication and knowledge management}''.

The adjective ``smart'' began to gain a increasingly notoriety between 2005 and 2007, when it started to be used to denote a sort of dream-city, i.e. a complex and optimized environment, or eco-system, where it could be desirable to live. It appeared immediately clear that the adjective smart was intended to go well beyond the meaning intelligent and/or to emphasize the use of IC and digital technologies\cite{giovannella_smart_2014-1}.

For this reason the authors chose to limit the search to articles published from 2005.

Smart cities are also a powerful ecosystem for learning. Smart city learning aim to support the improvement of all key factors contributing to the regional competitiveness: mobility, environment, people, quality of life and governance. The approach is aimed at optimizing resource consumption and saving time improving flows of people, goods and data\footnote{http://www.mifav.uniroma2.it/inevent/events/sclo/}.

Education in this context is pursued as a bottom-up process, where person and places are central. Smartness from a learning perspective exists both in the ambient data collected and among the communities that exists within a city.

The separation between student and teacher will fade out. Their role will be content or situation dependent: everybody will be a learner and the relation between persons will get a bigger role.

From the learning perspective, smart cities can be seen as an independent \textit{learning actor} that behaves like an autonomous entity which adapt itself in an evolving environment.

Despite this, the most interesting point of view for this work is probably the one that sees the smart city as a place where citizens learn smart-behaviors.

This scenario can involve traditional education which happens in facilities like schools and universities. The goal of this work is instead more oriented towards \textit{lifelong learning}, defined as the continuous build of skills to adapt and collaborate in dynamic ecosystems like smart cities.