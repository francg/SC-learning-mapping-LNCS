A significant portion of the articles dates back to 2005, this is probably connected to the momentum of the freshly introduced term of \textit{smart} city back in those years, like mentioned in section \ref{sec:intro}.
From the articles screened, some common concepts and research set-ups emerged.
For example several case studies involves schools, students and mobile devices.

A more technological-based approach, for example using custom-built hardware and prototypes, is almost never used.
From the methodological point of view, there seem to be a lack of Design Science\cite{hevner_three_2007} based approach, which would allow to focus more on the technology that fits best the desired learning objectives.