\subsection{Data Sources}
The articles were searched and collected using three different approaches:

\begin{enumerate}
\item keyword based search on different online databases
\item manual screening of selected conference proceedings
\item manual screening of selected journals
\end{enumerate}

The following online databases were used for the keyword based search: ISI Web of Science\footnote{\url{https://apps.webofknowledge.com/}}, ACM digital library\footnote{\url{https://dl.acm.org/}}, Elsevier - ScienceDirect\footnote{\url{https://www.sciencedirect.com/}}, Elsevier - Scopus\footnote{\url{http://www.scopus.com/}}, IEEE Xplore\footnote{\url{http://ieeexplore.ieee.org/}}.

\medskip

The following conference proceedings were searched for relevant articles:
\begin{itemize}
\item \textbf{CSCW} Computer-Supported Cooperative Work and Social Computing\footnote{\url{http://cscw.acm.org/}}
\item \textbf{CHI} Conference on Human Factors in Computing Systems\footnote{\url{http://chiYYYY.acm.org/}}
\item \textbf{EC-TEL} European Conference on Technology Enhanced Learning\footnote{\url{http://www.ec-tel.eu/}}
\item \textbf{AMI} International Joint Conference on Ambient Intelligence\footnote{\url{http://www.ami-conferences.org/}}
\item \textbf{C\&T} International Conference on Communities and Technologies\footnote{\url{http://comtech.community/}}
\end{itemize}

The following journal issues were searched for relevant articles:
\begin{itemize}
\item \textbf{IJDLDC} International Journal of Digital Literacy and Digital Competence vol. 3 n. 4 - Special Issue on \textit{``Smart City Learning, literacy and Competences''}\footnote{\url{http://www.igi-global.com/journal/international-journal-digital-literacy-digital/1170}}
\item \textbf{IxD\&A} Interaction Design and Architecture(s), vol. 16 (part I)\footnote{\url{http://www.mifav.uniroma2.it/inevent/events/idea2010/index.php?s=10&a=10&link=ToC_16_P}}, vol. 17 (part II)\footnote{\url{http://www.mifav.uniroma2.it/inevent/events/idea2010/index.php?s=10&a=10&link=ToC_17_P}} - Special Issue on \textit{``Smart City Learning - Visions and practical Implementations: toward Horizon 2020''} 
\end{itemize}

\subsection{Search and Keywords}
The keywords selection process was driven by the PICO framework. PICO helps to develop a comprehensive set of search keywords for quantitative research terms according to: Population, Intervention or Exposure (PECO), Comparison, Outcomes \cite{schardt_utilization_2007}.

Initially, keywords for all the sections of the framework were selected, but the authors decided later on to relax some constraints in order to avoid missing possible relevant articles. A \textit{context} section was also added to the schema.

Table \ref{table:keywords} shows the PICO(C) structure with associated keywords.

%% table of keywords here
\begin{table}[ht]
\renewcommand{\arraystretch}{1.3}
\centering
\caption{PICO(C) Driven Keywords Framing}
\label{table:keywords}
\begin{tabular}{|
>{\columncolor[HTML]{EFEFEF}}l |l|}
\hline
\textbf{Population} & \textit{-} \\ \hline
\textbf{Intervention} & \textit{learning} \\ \hline
\textbf{Comparison} & \textit{-} \\ \hline
\textbf{Outcome} & \textit{\begin{tabular}[c]{@{}l@{}}participation, collaboration, reflection,\\ awareness\end{tabular}} \\ \hline
\textbf{Context} & \textit{\begin{tabular}[c]{@{}l@{}}cities, smart city, urban, connected city,\\ intelligent city, digital city\end{tabular}} \\ \hline
\end{tabular}
\end{table}

A pilot search was conducted on some of the online databases in order to refine the keywords and find a search query that could be adapted and used in all the different online databases.

The final search query used is reported in Table \ref{table:query}.

%% search query
\begin{table}[ht]
\renewcommand{\arraystretch}{1.3}
\centering
\caption{Search Query}
\label{table:query}
\begin{tabular}{c|c}
\rowcolor[HTML]{EFEFEF} 
\textbf{Context} & \begin{tabular}[c]{@{}c@{}}(cities OR "smart city" OR urban\\ OR "connected city" OR "intelligent city"\\ OR "digital city")\end{tabular} \\
 & \textbf{AND} \\
\rowcolor[HTML]{EFEFEF} 
\textbf{Intervention} & ("learning") \\
 & \textbf{AND} \\
\rowcolor[HTML]{EFEFEF} 
\textbf{Outcome} & \begin{tabular}[c]{@{}c@{}}(participation OR collaboration\\ OR reflection OR awareness)\end{tabular}
\end{tabular}
\end{table}

Different online databases offer different levels of search functionalities and detail when going to use a complex query that possibly involves several keywords and fields (title, abstract, etc.).

Some of the difficulties encountered were:
\begin{itemize}
\item limit on the number of keywords that can be used;
\item limit on the fields where the search can be performed, search on title AND abstract not always possible;
\item no precise and direct control on the target search fields, keywords could be only searched on a preset aggregation of fields like title, abstract and article keywords;
\item different ways of coding the same logic expression, the same search string couldn't be reused on different databases;
\item different formats of the result set, in some cases was possible to batch-download the results, otherwise results were scraped using Zotero\footnote{\url{https://www.zotero.org/}} browser integration.
\end{itemize}

The keywords were searched on title and abstract when possible, otherwise only the abstract was used. In Table \ref{table:result_1} the size of the result sets for each online database is outlined.

The articles collected were imported in a Zotero library, and duplicates were manually removed.
The complete list of selected articles is available as a public online repository\footnote{\url{https://github.com/francg/IxD-A_SCL_systematic_mapping_articles}}.

%% table of result count
\begin{savenotes}
\begin{table}[ht]
\setlength{\tabcolsep}{8pt}
\centering
\caption{Result Set for online databases before duplicates removal}
\label{table:result_1}
\begin{tabular}{cccccc|c}
 & \cellcolor[HTML]{EFEFEF}\textbf{ISI} & \cellcolor[HTML]{EFEFEF}\textbf{ACM} & \cellcolor[HTML]{EFEFEF}\textbf{ScienceDirect} & \cellcolor[HTML]{EFEFEF}\textbf{Scopus} & \cellcolor[HTML]{EFEFEF}\textbf{IEEE} & \textbf{TOT} \\
\textit{n} & 938 & 35 & 162 & 1022 & 42 & 2199 \\
\textit{field} & topic\footnote{Topic fields include Titles, Abstracts, Keywords and Indexing fields such as Systematics, Taxonomic Terms and Descriptors} & abstract & abstract & abstract & abstract & 
\end{tabular}
\end{table}
\end{savenotes}

%% table no duplicates
\begin{table}[ht]
\setlength{\tabcolsep}{8pt}
\centering
\caption{Final Result Set without duplicates and selected articles for coding}
\label{table:result_2}
\begin{tabular}{cccc|c}
 & \cellcolor[HTML]{EFEFEF}\textbf{Online Databases} & \cellcolor[HTML]{EFEFEF}\textbf{IJDLDC} & \cellcolor[HTML]{EFEFEF}\textbf{IxD\&A} & \textbf{TOT} \\
\textit{n (no duplicates)} & 1485 & 5 & 11 & 1501 \\
\textit{selected} & 43 & 2 & 9 & 54
\end{tabular}
\end{table}